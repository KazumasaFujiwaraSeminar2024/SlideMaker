\documentclass[12pt,xelatex,ja=standard]{bxjsarticle}
%[[[ TeX setting
%[[[ package
\usepackage{zxjatype}
\usepackage{amsmath,amsthm}
\usepackage{unicode-math}
\usepackage{fontspec}
\usepackage{rgt}
\setmathfont{NewComputerModernMath}
%]]]

\begin{document}
\Nendo{2024}
\GakusekiNum{Y210005} % 学籍番号
\Myouji{石川}          % 自分の苗字
\Namae{拓海}          % 自分の名前
\ShidouMyouji{藤原}   % 指導教員の苗字
\ShidouNamae{和将}    % 指導教員の名前
\Shokukai{准教授}

\title{常微分方程式の初期値問題の解の一意性}
\begin{abstract}
\label{sec:intro}
大学の4年間で大学の数学を勉強してきた中で,
微分方程式の基礎的な解法に取り組んできたが,
理論を深く理解には至らなかった.
大学3年生で研究室配属になってから解析学の基礎的な知識を学び,
微分方程式の理解がより深くなり,
常微分方程式の初期値問題における解の一意性に興味を持つに至った.
本研究では, 常微分方程式の初期値問題の解の構成と解の一意性を証明を行う.
解を構成する方法はいくつか存在するが本論文は完備性を利用した大域解の一意性を示す.
本論文の構成としては,
第\ref{sec:積分作用素}章と
第\ref{sec:関数列}章で理論的な準備を行い,
第\ref{sec:微分方程式の初期値問題}章で解の存在と一意性を示す.
今回考える常微分方程式の初期値問題は次の形式で与えられるものとする:
	\[
	\frac{d}{dx}Y(x) = f(x, Y(x)),
	\quad Y(x_{0}) = Y_{0}.
	\]

第\ref{sec:積分作用素}章では,
Riemann積分の基本的性質についてまとめる.
Riemann積分は\eqref{eq:ode}を解析する為の,
対応する積分方程式を導入するために必要な基本的な概念である.
ここでは, Riemann積分の線型性と単調性について確認する.
また, 有界閉区間$I$上で,
ノルム$\|f\| = \sup_{x \in I} |f(x)|$を付加した,
連続関数の空間$C(I;\mathbb{R})$において,
積分作用素が連続線形作用素であることを示す

第\ref{sec:関数列}章では,
上記$C(I;\mathbb{R})$が完備である事を示す.
完備性は,積分方程式の解を構成するための重要な性質である事は次章で扱う.
加えて,
関数列の極限と積分の順序交換が可能であるための十分条件を求め,
$I$上の積分作用素が$C(I;\mathbb{R})$上の連続作用素であることを確認する.
この条件は, 逐次近似法を用いた積分方程式の解の構成において重要な役割を果たす.

第\ref{sec:微分方程式の初期値問題}章では,
上記問題の解の一意存在性を,
対応する積分方程式を介して検討する.
解の構成は,
Picardの逐次近似法を用いて行う.
具体的には,
連続関数の空間の完備性と積分作用素の有界性を利用して,
任意の有界閉区間において近似解の基本列を構成する事で,
解作用素の不動点として解が存在することを示す.
一意性の証明においては,
初期値を共有する2つの解の差をGronwallの不等式を用いて評価する.
\end{abstract}
\maketitle
\Mokuji
\section{はじめに}
本研究では, 常微分方程式の初期値問題の解の一意性を示すことが目的とする.
研究室配属になってから解析学の基礎的な知識を学び,
微分方程式の理解がより深くなり,
常微分方程式の初期値問題における解の一意性に興味を持つに至った.

本研究では, 以下の常微分方程式の初期値問題の解の一意存在性について考察する:
	\begin{align}
	\begin{cases}
	\frac{d}{dx}Y(x) = f(x, Y(x)), \quad x \in (a,b)\\
	\quad Y(a) = Y_{0},
	\end{cases}
	\label{eq:ode}
	\end{align}
ただし, $a < b$は実数である.
以降, $I = [a,b]$, $I^i=(a,b)$とする.
ここでいう解とは, 以下で与えられる古典解を指す:
\begin{Definition}
$Y ∈ C(I;\mathbb R) \cap C^1(I^i;\mathbb R)$
が$Y(a) = Y_0$を満たし,
$I^i$上で$\frac{d}{dx}Y(x) = f(x, Y(x))$を満たすとき,
$Y$を初期値問題\eqref{eq:ode}の解という.
\end{Definition}
また本論文では,
$f$は以下の意味でLipshitz連続であると仮定する.
\begin{Definition}[Lipshitz条件]
$\Omega=I\times \mathbb{R}$とする.
$Y \in C(I,\mathbb{R})$で,
$f(x,Y): I \times \mathbb{R} \to \mathbb{R}$とする.
$f$が$\Omega$においてLipshitz連続であるとは,
$(x,Y), (x,Z) \in \Omega$に対して,
	\[
	|f(x, Y) - f(x, Z)| \leq L |Y - Z|
	\]
が成り立つことである.
\end{Definition}
本論文の確認した主張は以下の通りである:
\begin{Theorem}
\label{Theorem:main}
$f$は$I \times \mathbb R$上でLipshitz連続であるとする.
このとき,
任意の$Y_{0} \in \mathbb{R}$に対して,
初期値問題\eqref{eq:ode}の解は$C(I,\mathbb{R})$上で一意存在する.
\end{Theorem}

本論文では$C(I,\mathbb{R})$の完備性と積分作用素の性質をもとに
定理\ref{Theorem:main}を証明する.

次章では,
積分の基本的性質を整理し, 積分作用素が連続関数の空間において有界線形作用素であることを示す.
\ref{sec:関数列}章では,
$C(I,\mathbb{R})$が下記に定義する一様ノルムに関して完備であることを示す.
\ref{sec:微分方程式の初期値問題}章では,
Picardの逐次近似法を用いて定理\ref{Theorem:main}を証明する.

\section{積分作用素}
\label{sec:積分作用素}
この章では,
\cite{R1,R2}に基づいて,
Riemann積分の基本的性質についてまとめる.
なお, 実数$a>b$に対して, $I=I$を閉区間とする.

\begin{Definition}[Riemann積分]
$f$を閉区間$I$で有界な関数とする.
閉区間$I$で分割を$\Delta:a=x_{0}\leq x_{1} \leq x_{2} \leq x_{3} \cdot \cdot \cdot  \leq x_{n}=b$と定義する.この時,$d(\Delta)$を以下のように定義する
\[d(\Delta)=\max\{x_{k}-x_{k-1}|k=1,2,3,\cdot \cdot\cdot n \}\]
$\xi_{k}$を$x_{k-1} \leq \xi_{k}\leq x_{k}$を満たすとするとき,
$\displaystyle \lim_{\substack{d(\Delta) \to 0}} \sum_{k=1}^{n} f(\xi_{k}) \ (x_k - x_{k-1})$の極限が存在するとき,Riemann積分可能と定義する.その極限を$\displaystyle \int_{a}^{b} f(x) dx$とする.
\end{Definition}

\begin{Lemma}[定積分の基本的性質]
$I$上Riemann可積分な$f,g :I → ℝ$は,
以下の性質を満たす:
	\begin{align*}
	\int_{a}^{b} (f(t) + g(t)) \, dt
	&= \int_{a}^{b} f(t) \, dt + \int_{a}^{b} g(t) \, dt,\\
	\int_{a}^{b} c f(t) \, dt
	&= c \int_{a}^{b} f(t) \, dt \quad ( c \in \mathbb R)\\
	\bigg| \int_{a}^{b} f(t) \, dt \bigg|
	&\leq \displaystyle \int_{a}^{b} |f(t)| \, dt.
	\end{align*}
\end{Lemma}

また, 連続関数に おいては, 次のような性質が成立する.

\begin{Lemma}
閉区間で連続な関数は積分可能である
\end{Lemma}

連続関数に対する積分の性質を調べるために,
ノルム空間を定義する.

\begin{Definition}[ノルム空間]
線形空間 $X$ において実数値関数 $\|\cdot\|: X \to \lbrack 0, \infty)$
が次の4つの条件を満たすとき,
$\|\cdot\|$ を $X$ 上のノルムといい,
$(X,\|\cdot\|)$ を \textbf{ノルム空間}(normed space)と呼ぶ.
\begin{itemize}
\item $\|f\| \geq 0$, $\|f\| = 0 \Leftrightarrow f = 0$
\item $\|kf\| = |k| \, \|f\|$
\item $\|f+g\| \leq \|f\| + \|g\|$
\end{itemize}
\end{Definition}

\begin{Definition}[一様ノルム]
$I$上有界な関数$f$に対して,
$\|f\|_{L^\infty(I)}$を次のように定義する:
	\[
	\|f\|_{L^\infty(I)} = \sup_{x \in I} |f(x)|.
	\]
\end{Definition}

\begin{Proposition}
$(C(I,\mathbb{R}), \|\cdot\|_{L^\infty(I)})$はノルム空間である.
\end{Proposition}

\begin{Theorem}
$f \in C(I, \mathbb R)$に対して,
	\[
	T(f) = \int_a^{\cdot} f(t) \, dt
	\]
とすると,
$T(f) \in C(I, \mathbb R)$である.
特に, $T$は$C(I, \mathbb R)$上の連続な線型作用素である.
\end{Theorem}
\begin{proof}
積分の線形性により線形性が従う.

$f,g \in C(I, \mathbb R)$に対して,
	\begin{align*}
	\bigg\| \int_0^{\cdot} (f(t) - g(t)) \, dt \bigg\|_{L^\infty(I)}
	&\leq \bigg\|
		\int_0^{\cdot} \|f - g\|_{L^\infty(I)} \, dt
	\bigg\|_{L^\infty(I)}\\
	&\leq |b-a| \|f - g\|_{L^\infty(I)}
	\end{align*}
から連続性が従う.
\end{proof}

\section{$(C(I,\mathbb{R}),\| \|_{L^\infty(I)})$の完備性}
\label{sec:関数列}
またこの章では,
$C(I,\mathbb{R})$での関数列の収束について考察する.

\begin{Definition}[一様収束の定義]
$I$上で定義された有界な実数値関数の列$\{{f_{n}\}}_{n=1}^{\infty}$
が$f$に$I$上で一様収束するとは,
	\[
	\lim_{n \to \infty} \|f_n - f \|_{L^\infty(I)} = 0
	\]
の事をいう.
\end{Definition}

\begin{Lemma}[Cauchyの判定法]
\label{Lemma:Cauchy}
$\{f_{n}\}_{n=1}^{\infty}$をI上で定義された関数列とし,
$s_{n}=\displaystyle \sum_{i=1}^{n} f_i$とする.
$s_{n}$が$I$ 上で一様収束するための必要十分条件は,
任意の $\epsilon > 0$ に対してある番号 $N$ があって,
$m, n > N$ に対し
	\begin{align}
	\| s_n - s_m \|_{L^\infty(I)} < \epsilon
	\label{eq:Cauchy}
	\end{align}
が成り立つことである.
\end{Lemma}
\begin{proof}
($\Rightarrow$)
一様収束の定義より任意の$ε > 0$ に対してある自然数$N$が存在して,
$n ≥ N$ に対して
$\|f_n - f\|_{L^\infty(I)} < \frac{\varepsilon}{2}$
が成立する.
よって, $n, m \geq N$, であれば
	\[
	|s_{n}(x) - s_{m}(x)|
	\leq |s_{n}(x) - s(x)| + |s(x) - s_{m}(x)| < ε
	\]
より, ${s_n}_{n = 1}^\infty$は
$(C(I,\mathbb{R}),\|\cdot\|_{L^\infty(I)})$の基本列である.

$(\Leftarrow)$ を示す.各 $x \in I$について,
$\{s_{n}(x)\}_{n=1}^{\infty}$
は実数の基本列であるので$\lim_{n \to \infty} s_{n}(x) = s(x)$なる
関数$s: I \to \mathbb R$ が存在する.
\eqref{eq:Cauchy}より, 任意の $x \in I$ に対して,
	\[
	|s_n(x) - s(x)|
	\leq \liminf_{m \to \infty}
	(|s_n(x) - s_m(x)| + |s_m(x) - s(x)|) < ε
	\]
である.
従って, 両辺$x$について上限を取ると,
	\[
	\|s_n - s\|_{L^\infty(I)} < ε
	\]
であるから, 主張を得る.
\end{proof}

この命題を用いると, 次の判定法が従う.

\begin{Theorem}[WeirstraßのM-判定法]
$\{f_{n}\}_{n=1}^{\infty}$をI上で定義された関数列とし,
	\[
	\|f_{n}\|_{L^\infty(I)} \leq M_{n}
	\]
がどの$n$でも成立するとする.
$\sum_{n=1}^{\infty} M_{n} < \infty $であるとき,
関数項級数$\{ \sum_{k=1}^{n} f_{k} \}^{\infty}_{n=1}$は$I$で一様収束する.
\end{Theorem}

\begin{proof}
$\{ M_n \}_{n=1}^{\infty}$の級数は収束するので,
$s_n = \sum_{k=1}^{n} M_k$ とすると, $s_n$は基本列である.
	\begin{align*}
	\bigg\| \sum_{k=m}^n f_k \bigg\|_{L^\infty(I)}
	&\leq \sum_{k=m}^n \| f_k \|_{L^\infty(I)}\\
	&\leq \sum_{k=m}^n M_k = s_n - s_{m-1}
	\end{align*}
であるから, $\{ \sum_{k=m}^n f_k \}_{n=1}^{\infty}$は
$\| \cdot \|_{L^\infty(I)}$に関して基本列であるので,
補題\ref{Lemma:Cauchy}より一様収束する.
\end{proof}

また次の連続関数に対する性質も確認する.

\begin{Lemma}
$\{f_n(x)\}_{n=1}^{\infty} \subset C(I;\mathbb R)$が,
$f$に$I$上で一様収束するならば,
$f \in C(I;\mathbb R)$である.
\end{Lemma}

\begin{proof}
任意の $\varepsilon > 0$ に対して,
ある自然数 $N$ が存在し, $n \geq N$ に対して
$\| f_n - f \|_{L^\infty(I)} < \frac{\varepsilon}{3}$ が成り立つ.

一方, $f_N$は$I$で連続であるため,
$a \in I$に対して, $\delta_a > 0$ が存在し, 次が成り立つ:
	\[
	| x - a | < \delta_a
	\Rightarrow | f_N(x) - f_N(a) | < \frac{\varepsilon}{3}
	\]

以上のことから, 次のように評価できる:
	\[
	| x - a | < \delta_a
	\Rightarrow | f(x) - f(a) |
	\leq 2 \| f - f_N \|_{L^\infty(I)}
	+ | f_N(x) - f_N(a) | < \varepsilon
	\]
したがって, $f \in C(I;\mathbb R)$である.
\end{proof}

すなわち, 次の様に上記を纏めることができる.

\begin{Corollary}
$(C(I,\mathbb{R}),\| \cdot \|_{L^\infty(I)})$は完備である.
\end{Corollary}

\section{主定理の証明}
\label{sec:微分方程式の初期値問題}
定理\ref{Theorem:main}の証明を行うにあたって,
先験的に\eqref{eq:ode}の両辺を積分$(x_{0},x)$で積分すると
	\begin{align}
	Y(x)=Y_{0}+\displaystyle \int_{x_{0}}^{x} f(t,Y(t)) dt
	\label{eq:integral}
	\end{align}
が得られる.
積分方程式\eqref{eq:integral}を満足する$I$上の連続関数$Y$が一意存在すれば,
微積分学の基本定理より$Y$は$C(I,\mathbb{R})$上で微分可能であり,
微分方程式\eqref{eq:ode}の解である事が直ちに従う.
この章では,
積分方程式\eqref{eq:integral}の解を
$C(I,\mathbb{R})$の完備性を用いて構成する.
そして, \eqref{eq:integral}の解が\eqref{eq:ode}の解であり,
$C(I,\mathbb{R})$上で一意的であることを示す.

次のGronwallの不等式は,
解の一意性を示すために重要な役割を果たす.
\begin{Lemma}[Gronwallの不等式]
$h,g$を区間$[x_{0},T]$ $(x_{0}<T\leq +\infty)$上の関数とし,
$g\cdot h,g$は$[x_0,T]$上で可積分かつ,
$g$は非負値関数とする.
この時次の関係がある:
	\[
	h(x) \leq A + \int_{x_0}^x g(\xi) h(\xi) \, d\xi \implies
	h(x) \leq A \exp\left(\int_{x_0}^x g(\xi) \, d\xi\right)
	\]
\end{Lemma}

\begin{proof}
$H(x)=A+\int_{x_0}^x g(\xi) h(\xi)$とする
$h(x)\leq H(x)$, $H'(x)=g(x)h(x)$である事に注意する.
両辺$g(x)$をかけると$g(x)h(x)\leq g(x)H(x)$となり,
この両辺に$\exp\left(-\int_{x_0}^x g(\xi) \, d\xi\right)$をかけて整理すると
	\begin{align*}
	&\frac{d}{dx} \bigg( \exp \bigg( -\int_{x_0}^x g(\xi) \, d\xi \bigg) H(x)\bigg)\\
	&=H(x)' \exp \bigg( -\int_{x_0}^x g(\xi) \, d\xi \bigg)
	-g(x)H(x) \exp \bigg( - \int_{x_0}^x g(\xi) \, d\xi \bigg)
	\leq 0
	\end{align*}
が得られる.
従って,$\exp(-\int_{x_0}^x g(\xi) \, d\xi)H(x) \leq H(x_{0})=A$が成立する
ので題意を得る.
\end{proof}

\begin{proof}[定理\ref{Theorem:main}の証明]
$n \geq 1$に対して,
	\[
	Y_{n}(x)=Y_{0}+\displaystyle \int_{a}^{x} f(t,Y_{n-1}(t)) dt
	\]
とする. ただし, $Y_{0}(x) \equiv Y_0$とした.
ここに,
$\{Y_{n}\}_{n=1}^{\infty} ⊂ C(I)$は$J = [a,c] ⊂ I$上で一様収束していて,
かつ一様収束極限$Y$は$J$上で上記積分方程式の解であることを示す.

$M = \sup_{x \in J} |f(x, Y_0)|$
にたいして,
評価
	\[
	| Y_{n}(x) - Y_{n-1}(x) |
	\leq M L^{n-1} \frac{|x - a|^{n}}{n!}, \quad x \in J,
	\]
が成り立つことを示す.
実際, $n=1$の場合,
	\[
	\left| Y_{1}(x)-Y_{0}(x) \right|
	=\left|\displaystyle \int_{a}^{x} f(\xi,Y_{0}(\xi)) d\xi
	\right|\leq M|x-a|
	\]
となるから成立.
$n=k$に対して上記不等式が成立すると仮定する.
$n=k+1$に対して,
	\begin{align*}
	\left| Y_{k+1}(x)-Y_{k}(x) \right|
	&=  \left| \int_{a}^{x} f(\xi,Y_{k}(\xi)) d\xi
	- \int_{a}^{x}  f(\xi,Y_{k-1}(\xi)) d\xi\right|\\
	&\leq \int_{a}^{x} \left|f(\xi,Y_{k}(\xi))-f(\xi,Y_{k-1})\right| d\xi\\
	&\leq \int_{a}^{x} L\left|Y_{k}(\xi))-Y_{k-1}(\xi))\right| d\xi\\
	&\leq \int_{a}^{x}ML^{k} \frac{|\xi - a|^{k}}{k!} d\xi\\
	&\leq ML^{k-1}\frac{|x - a|^{k+1}}{(k+1)!}
	\end{align*}
従って主張を得る.

	\[
	\sum_{n=1}^{\infty}ML^{k-1} \frac{|x - a|^{k+1}}{k!}
	\leq \sum_{n=1}^{\infty} M L^{k-1}\frac{|J|^{k}}{k!}
	=M \cdot \frac{e^{|J|}-1}{L}
	\]
であるから,
Weierstraßの$M-$判定法により,
$Y_{n}=Y_{0}+\sum_{k=1}^{n} [Y_{k}(x)-Y_{k-1}(x)]$
は一様収束する.
この一様収束極限を$Y$と置く.
連続関数列の一様収束極限関数は連続であることより,
$Y ∈ C(J, \mathbb R)$である.

第三章で述べた積分と極限との交換の十分条件により
	\[
	\lim_{n \to \infty} \int_{a}^{x} f(\xi,Y_{n}(\xi)) \, d\xi
	= \int_{a}^{x} f(\xi,Y(\xi)) \, d\xi
	\]
である.
従って, $Y$は$J$上で積分方程式\eqref{eq:integral}の解である.
	\begin{align*}
	Y(x) &= Y_{0} + \int_{a}^{x} f(\xi, Y(\xi)) \, d\xi,\\
	Z(x) &= Y_{0} + \int_{a}^{x} f(\xi, Z(\xi)) \, d\xi
	\end{align*}
を積分方程式\eqref{eq:integral}の解とする.
$x \in J$に対して,
	\begin{align*}
	|Y(x) - Z(x)|
	&= \bigg| \int_{x_{0}}^{x} \big( f(\xi, Y(\xi)) - f(\xi, Z(\xi)) \big) \, d\xi \bigg|\\
	&\leq \int_{x_{0}}^{x} L |Y(\xi) - Z(\xi)| \, d\xi
	\end{align*}
である.
従って, Gronwallの不等式より, $Y$と$Z$は一致する.
\end{proof}

\begin{thebibliography}{9}
\bibitem{R1}
鈴木武, 柴田良弘, 山田義雄, 田中和永, 理工系の微分積分1, 内田老鶴圃, 2007

\bibitem{R2}
鈴木武, 柴田良弘, 山田義雄, 田中和永, 理工系の微分積分2, 内田老鶴圃, 2007

\bibitem{O1}
大谷光春, 常微分方程式概論, サイエンス社, 2011
\end{thebibliography}

\end{document}
