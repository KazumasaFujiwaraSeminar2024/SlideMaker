% !TEX program = lualatex
\documentclass[aspectratio=169]{beamer}
\usepackage{luatexja}
\usepackage{amsmath,amssymb}
\usepackage{graphicx}
\usepackage{hyperref}
\usepackage{fancybox}
\usepackage{tcolorbox}
\usetheme{Copenhagen}
\tcbuselibrary{skins}

\setbeamertemplate{navigation symbols}{}
\setbeamertemplate{footline}[frame number]
\setbeamertemplate{headline}{}

\setbeamertemplate{frametitle}{%
  \vspace*{-0.2cm}%
  \begin{beamercolorbox}[wd=\paperwidth,ht=1.2cm,dp=0.3cm,leftskip=0pt,rightskip=0pt]{frametitle}
    \hspace*{1em}\Huge\insertframetitle
  \end{beamercolorbox}
}

\newenvironment{sectionblock}[1]{%
  \begin{minipage}{\textwidth}%
    \textbf{\large #1}\par\vspace{0.5em}%
}{%
  \end{minipage}\vspace{1em}%
}

\newcommand{\sectioncontent}[2]{%
  \begin{sectionblock}{#1}%
    \begin{itemize}%
      #2%
    \end{itemize}%
  \end{sectionblock}%
}

\begin{document}

\begin{frame}
\frametitle{証明の概略1}
\sectioncontent{結論の要点}{
  \item 常微分方程式の初期値問題の解の一意性を完備性に基づいて証明する。
  \item 証明には関数空間の完備性および縮小写像の原理を利用。
  \item 本研究の目標は、大域的な一意性の構成的な理解。
}
\end{frame}

\begin{frame}
\frametitle{証明の概略2}
\sectioncontent{使用する定義式}{
  \item 初期値問題:$\displaystyle \frac{d}{dx}Y(x) = f(x, Y(x)), \quad Y(x_0) = Y_0$
  \item 積分形式:$\displaystyle Y(x) = Y_0 + \int_{x_0}^{x} f(t, Y(t)) \, dt$
  \item 関数空間:$C([x_0 - a, x_0 + a])$(連続関数全体の空間)
  \item 距離:$d(Y_1, Y_2) = \sup_{x \in I} |Y_1(x) - Y_2(x)|$
}
\end{frame}

\begin{frame}
\frametitle{証明の概略3}
\sectioncontent{計算の要点と意義}{
  \item 上記積分式は関数空間上の作用素として定義される。
  \item その作用素が縮小写像であることを示し、Banachの不動点定理を適用。
  \item よって、初期値問題は一意な解を持つ。
  \item 完備性の意義:収束先が必ず空間内に存在し、数学的保証を与える。
}
\end{frame}

\end{document}
