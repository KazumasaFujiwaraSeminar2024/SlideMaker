\documentclass[a4paper]{article}
\usepackage{tcolorbox}
\tcbuselibrary{breakable, skins, theorems}
\usepackage{luatexja}            % LuaTeX用の日本語パッケージ
\usepackage{fontspec}
\usepackage{amsmath}
\usepackage{amsthm}
\usepackage{unicode-math}
\setmainfont{IPAexMincho}       % 日本語フォントを指定
\usepackage{xcolor}              % 色の定義用パッケージ
\usepackage{listings}            % コード表示用パッケージ
\newtheorem{theorem1}{定義}
\newtheorem{theorem2}{定理}

\begin{document}

    \title{テイラー展開}
    \author{松本 侑万}
    \date{2025}
    \maketitle

    \begin{tcolorbox}[
        colback=white,
        colframe=green!35!black,
        fonttitle=\bfseries,
        breakable=true]
        \begin{theorem1}
            関数 $f$ を $C^\infty$ 級関数とする(すなわち無限回微分可能な関数)。$f$ が点 $a$ の近くで
            \[
            f(x) = \sum_{n=0}^\infty \frac{f^{(n)}(a)}{n!} (x - a)^n \\
            = f(a) + f'(a)(x - a) + \frac{f''(a)}{2!}(x - a)^2 + \frac{f'''(a)}{3!}(x - a)^3 + \frac{f''''(a)}{4!}(x - a)^4 + \cdots
            \]
            とできるとき、これを $f$ の $a$ の周りのテイラー展開(テーラー展開; Taylor expansion)といい、この級数をテイラー級数 (Taylor series) という。

            定義域全ての点 $a$ において、その十分近くでテイラー展開可能な関数を解析的 (analytic) であるという。
        \end{theorem1}
    \end{tcolorbox}
    
    \begin{tcolorbox}[
        colback=white,
        colframe=green!35!black,
        fonttitle=\bfseries,
        breakable=true]
        \begin{theorem2}
            関数 $f$ が点 $a$ の近くで $n$ 回微分可能であるとする。このとき、テイラー展開は次のように与えられる:
            \[
            f(x) = \sum_{k=0}^{n-1} \frac{f^{(k)}(a)}{k!} (x-a)^k + \frac{f^{(n)}(c)}{n!} (x-a)^n
            \]
            ここで、$f^{(k)}(a)$ は $a$ での $k$ 次の導関数を表し、最終項 $\frac{f^{(n)}(c)}{n!} (x-a)^n$ はラグランジュの剰余項 $R_n$ として知られ、$a < c < x$ である $c$ が存在する。展開は以下の形式で表される:
            \[
            f(x) = f(a) + f'(a)(x-a) + \frac{f''(a)}{2!}(x-a)^2 + \cdots + \frac{f^{(n)}(c)}{n!}(x-a)^n
            \]
        \end{theorem2}
    \end{tcolorbox}
\end{document}