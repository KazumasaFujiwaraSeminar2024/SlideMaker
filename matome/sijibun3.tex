\documentclass[dvipdfmx]{beamer} % DVIドライバは使わないので dvipdfmxを外すなら問題なし
\usepackage{luatexja}
\usepackage{amsmath}
\usepackage{graphicx}
\usepackage{booktabs}
\usepackage{tikz}
\usepackage{fontspec}
\setmainfont{IPAexMincho} % 日本語対応のフォント

\usetheme{Madrid}
\usecolortheme{default}

\title{文書間類似度とベクトル距離の関係に関する検討}
\author{〇〇大学 〇〇学部 〇〇 〇〇}
\date{\today}

\begin{document}

\begin{frame}[fragile]{研究概要}
\small
\begin{block}{背景}
情報検索における類似文書の判定は、検索精度や推薦精度を大きく左右する重要な課題である。
\end{block}

\vspace{0.4em}
\begin{block}{目的}
文書ベクトルの距離と文書間類似度の関係を調査し、より精度の高い類似度推定手法を提案する。
\end{block}

\vspace{0.4em}
\begin{block}{方法}
TF-IDFやBERTによる文書ベクトルを用い、コサイン類似度やユークリッド距離との関係を分析した。また、クラスタリングによる分布可視化も行った。
\end{block}

\vspace{0.4em}
\begin{block}{結果}
距離尺度と類似度の間に非線形な相関が見られ、BERTベースの手法が高い再現率を示した。
\end{block}

\vspace{0.4em}
\begin{block}{結論・意義}
本研究は距離と類似度の関係性の再評価を促し、類似文書検索における性能改善の一助となることを示した。
\end{block}
\end{frame}

\end{document}
