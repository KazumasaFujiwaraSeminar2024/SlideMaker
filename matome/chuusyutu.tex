\documentclass[a4paper,12pt]{article}
\usepackage{fontspec}
\usepackage{amsmath, amssymb}
\usepackage{geometry}
\geometry{left=25mm,right=25mm,top=30mm,bottom=30mm}

\setmainfont{IPAexMincho} % 日本語のためのフォント設定、必要に応じて変更可能

\title{常微分方程式の初期値問題の解の一意性}

\begin{document}
\maketitle

\begin{enumerate}
  \item \textbf{解の定義}
  \begin{itemize}
    \item 初期値 \( Y(a) = Y_0 \) を満たし、\( \frac{dY}{dx} = f(x, Y(x)) \) を区間 \( I^i = (a, b) \) 上で満たす関数 \( Y \)\\
  \end{itemize}

  \item \textbf{Lipshitz 条件}
  \begin{itemize}
    \item 関数 \( f \) がLipshitz連続である条件についての説明\\
  \end{itemize}

  \item \textbf{ノルム空間と一様ノルム}
  \begin{itemize}
    \item ノルム空間の定義と \( C(I, \mathbb{R}) \) の完備性\\
  \end{itemize}

  \item \textbf{積分作用素とその連続性}
  \begin{itemize}
    \item \( T(f) \) が連続な線形作用素であることの説明\\
  \end{itemize}

  \item \textbf{一階常微分方程式の解の構成}
  \begin{itemize}
    \item Picardの逐次近似法による解の構成と一様収束の証明\\
  \end{itemize}

  \item \textbf{解の一意性の証明}
  \begin{itemize}
    \item Gronwallの不等式を用いた解の一意性の証明\\
  \end{itemize}

  \item \textbf{参考文献}
  \begin{itemize}
    \item 主な参考文献のリスト\\
  \end{itemize}

  \item \textbf{補足資料}
  \begin{itemize}
    \item 一様収束、Cauchyの判定法、WeirstraßのM-判定法についての補足説明\\
  \end{itemize}

\end{enumerate}

\end{document}
