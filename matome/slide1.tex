\documentclass{article}
\usepackage{luatexja}            % LuaTeX用の日本語パッケージ
\usepackage{luatexja-fontspec}   % フォント指定用パッケージ
\setmainjfont{IPAexMincho}       % 日本語フォントを指定
\usepackage{xcolor}              % 色の定義用パッケージ
\usepackage{listings}            % コード表示用パッケージ
\usepackage{graphicx}

\lstset{
    basicstyle=\ttfamily,
    frame=single,
    breaklines=true,
    numbers=left,
    numberstyle=\tiny,
    backgroundcolor=\color{gray!10},
    rulesepcolor=\color{gray!10}
}

\lstdefinelanguage{json}{
    basicstyle=\normalsize\ttfamily,
    numbers=left,
    numberstyle=\scriptsize,
    stepnumber=1,
    numbersep=8pt,
    showstringspaces=false,
    breaklines=true,
    frame=lines,
    backgroundcolor=\color{gray!10},
    stringstyle=\color{red},
    keywordstyle=\color{blue}\bfseries,
    keywords={true, false, null}
}

\begin{document}

\part{インフルエンザが”最も早い流行”...熱が出ない”隠れインフル”に要注意!感染症3つ同時流行で”トリプルデミック”も}

厳しい寒暖差で体調不良を起こしやすい中、2024年11月12日、厚生労働省はインフルエンザが全国的な竜後期に入ったと発表。

\begin{figure}[h]
    \centering
    \includegraphics[width=\textwidth]{スクリーンショット 2025-01-09 162524.png}
    \caption{こちらの画像のキャプション}
    \label{fig:my_label}
\end{figure}

1医療機関あたりの患者数は、流行の目安である「1」を超えた「1.04」人となりました。

東京・大田区にある「竹内内科小児科医院」には、1日約30人がせきや熱の症状を訴え、発熱外来を訪れました。

そのうち、約3割は「マイコプラズマ肺炎」と診断されていますが、11月に入ってからは「インフルエンザ」診断される人も増えているといいます。

”隠れインフル”かどうか気がつくための”サイン”として、以下のようなものがあります。

\begin{enumerate}
\item 倦怠感、関節痛、せき、のどの痛み、鼻水
\item つらくて食欲がない
\item 息苦しく寝られない
\item 尿が出ない

熱はなくとも、これらの症状はでたときは、インフルエンザを疑って早めに病院を受信してください。