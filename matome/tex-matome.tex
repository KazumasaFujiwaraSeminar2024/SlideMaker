\documentclass{article}
\usepackage{luatexja}            % LuaTeX用の日本語パッケージ
\usepackage{luatexja-fontspec}   % フォント指定用パッケージ
\setmainjfont{IPAexMincho}       % 日本語フォントを指定
\usepackage{xcolor}              % 色の定義用パッケージ
\usepackage{listings}            % コード表示用パッケージ

\lstset{
    basicstyle=\ttfamily,
    frame=single,
    breaklines=true,
    numbers=left,
    numberstyle=\tiny,
    backgroundcolor=\color{gray!10},
    rulesepcolor=\color{gray!10}
}

\lstdefinelanguage{json}{
    basicstyle=\normalsize\ttfamily,
    numbers=left,
    numberstyle=\scriptsize,
    stepnumber=1,
    numbersep=8pt,
    showstringspaces=false,
    breaklines=true,
    frame=lines,
    backgroundcolor=\color{gray!10},
    stringstyle=\color{red},
    keywordstyle=\color{blue}\bfseries,
    keywords={true, false, null}
}

\begin{document}

\part{wslでtexliveをinstallする}

\section{TeXLiveをインストール}
Ubuntu/Debian系の場合、ターミナルで以下を実行する。\par

\begin{lstlisting}[language=bash]
sudo apt update
sudo apt install texlive-full
\end{lstlisting}

Windowsの場合\par
\begin{enumerate}
\item TeXLiveの公式サイトからインストーラーをダウンロードする。\par
\item インストーラーを実行し、指示にしたがってインストールする。\par
\end{enumerate}

\par
macOSの場合、以下を実行する。\par

\begin{lstlisting}[language=bash]
brew install --cask mactex
\end{lstlisting}

\section{VS Codeの拡張機能をインストール}
VS CodeでLaTeXを快適に使うために、以下の拡張機能をインストールする。\par

\begin{enumerate}
\item LaTeX Workshopをインストールする\par
\item VS Codeを開く\par
\item 左側の拡張機能アイコンをクリック\par
\item 検索バーに「LaTeX Workshop」と入力\par
\item「LaTeX Workshop」を選択し、「インストール」をクリック\par
\end{enumerate}

\newpage

\section{TeXLiveとVS Codeの設定}
LaTeX Workshop拡張機能を適切に動作させるために設定を確認します。\par

\begin{enumerate}
    \item 設定ファイルを編集\par
    以下の内容を`Settings (JSON)`に追加します。\par
    
    \begin{lstlisting}[language=json]
    "latex-workshop.latex.toolchain": [
        {
            "command": "pdflatex",
            "args": [
                "-synctex=1",
                "-interaction=nonstopmode",
                "-file-line-error",
                "%DOC%"
            ]
        }
    ],
    "latex-workshop.latex.outDir": "./out"
    \end{lstlisting}
    
    \item 必要に応じて環境変数を確認\par
    以下を実行して、PATHが正しいことを確認します。\par
    
    \begin{lstlisting}[language=bash]
    echo $PATH
    \end{lstlisting}
\end{enumerate}

\section{動作確認}

\begin{enumerate}
    \item `tex`ファイルを作成\par

    \begin{lstlisting}[language=tex]
    \documentclass{article}
    \begin{document}
    Hello, LaTeX!
    \end{document}
    \end{lstlisting}
    \item ファイルを保存し、自動でPDFが生成されるか確認します。\par
\end{enumerate}

\newpage

\part{rgtレポジトリをクローンし、rgt.styをtexmf-homeに設置する}

\section{レポジトリをクローン}
以下を実行してレポジトリをクローンします。\par

\begin{lstlisting}[language=bash]
git clone https://github.com/username/rgt.git
\end{lstlisting}

\section{クローン後の確認}

\begin{enumerate}
    \item ディレクトリに移動し、ファイル一覧を確認します。\par

    \begin{lstlisting}[language=bash]
    cd rgt
    ls
    \end{lstlisting}
    
    \item 出力された結果からどのディレクトリにクローンされているかがわかります。\par
    \item \begin{lstlisting}[language=bash]
        README.md doc example
        \end{lstlisting}

\end{enumerate}



\section{rgt.styをtexmf-homeに設置}
以下を実行して、`texmf-home`にパスが通っていることを確認します。\par

\begin{lstlisting}[language=bash]
mktexlsr
kpsewhich rgt.sty
\end{lstlisting}

    \part{多重texmfツリー}

    \section{多重texmfツリーとは}
    異なる種類のTeXパッケージやフォントなどを整理し、管理するためのディレクトリ構造のこと。

    \section{基本的なtexmfツリーの構成}

        \begin{enumerate}

            \item メインのtexmfツリー
                \begin{itemize}
                    \item TeX LiveやMiKTeXなどのTeXディストリビューションに含まれる標準のツリー
                    \item グローバルなパッケージやフォントがインストールされる
                \end{itemize}
            
            \item ローカルtexmfツリー   
                \begin{itemize}
                    \item ユーザが追加のパッケージやフォントをインストールするためのツリー
                    \item システムのメインツリーに影響を与えることなく、個別のカスタマイズが可能
                \end{itemize}

            \item ホームtexmfツリー
                \begin{itemize}
                    \item 個々のユーザーのホームディレクトリ内に設定されるツリー
                    \item ユーザー固有の設定やパッケージを保持
                \end{itemize}

        \end{enumerate}

    \section{多重texmfツリーの利点}

        \begin{enumerate}

            \item 柔軟性
                \begin{itemize}
                    \item 異なるユーザーが異なる設定やパッケージを使用できるため、複数人での作業が効率的に行える
                \end{itemize}

            \item 安全性
                \begin{itemize}
                    \item システムワイドな設定を変更することなく、ユーザーが自由にパッケージを追加・更新できる
                \end{itemize}

            \item 整理整頓
                \begin{itemize}
                    \item パッケージを目的別に分類することで、管理がしやすくなる
                \end{itemize}

        \end{enumerate}

\end{document}
