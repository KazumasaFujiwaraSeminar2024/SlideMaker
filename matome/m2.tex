\documentclass[a4paper]{article}
\usepackage{luatexja}            % LuaTeX用の日本語パッケージ
\usepackage{fontspec}
\setmainfont{IPAexMincho}       % 日本語フォントを指定
\usepackage{xcolor}              % 色の定義用パッケージ
\usepackage{listings}            % コード表示用パッケージ

\begin{document}

    \title{お年玉の歴史}
    \author{松本 侑万}
    \date{2025}
    \maketitle

    \section{お年玉の歴史と文化}
    お年玉は、新年を祝い、子供たちの成長と幸福を願うために贈られる日本の伝統的な習慣です。\\
    この習慣は、もともとは江戸時代に確立されたとされていますが、その起源はもっと古くからあります。\\

    \section{起源と変遷}
    お年玉の起源には複数の説がありますが、最も広く受け入れられているのは、新年に歳神様に供えられた鏡餅を分け与えることから始まったという説です。\\
    当初はお金ではなく、餅やその他の食べ物が贈られていました。ゲンダイにおいては、主に現金が贈られるようになっていますが、その額は子供の年齢や親族の間の関係によって異なります。\\

    \section{世界の類似文化}
    お年玉と似た文化は世界中に存在します。例えば、中国では「紅包(ホンバオ)」と呼ばれる赤い封筒にお金を入れて贈る習慣があります。\\
    これは春節(旧正月)に行われ、幸運と繁栄を願う意味が込められいます。韓国では「セベットン」という名で、新年のあいさつと共にお金が贈られます。これらの習慣も家族や親戚間での絆を強化し、新年の祝福を象徴しています。\\

    \section{現代のお年玉}
    現代では、お年玉を贈る際のマナーや形式にも変化が見られます。お礼はピン札で用意することが好まれることが多いですが、最近ではその必要性は少し減少しています。また、デジタルギフトやプロペイドカードなど、現金以外の形でも贈ることが一般的になりつつあります。\\
\end{document}