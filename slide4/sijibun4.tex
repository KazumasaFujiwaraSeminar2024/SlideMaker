% !TEX program = lualatex
\documentclass[aspectratio=169]{beamer}
\usepackage{luatexja}
\usepackage{amsmath,amssymb}
\usepackage{graphicx}
\usepackage{hyperref}
\usepackage{fancybox}
\usepackage{tcolorbox}
\usetheme{Copenhagen}
\tcbuselibrary{skins}

% スライドのスタイル設定
\setbeamertemplate{navigation symbols}{}
\setbeamertemplate{footline}[frame number]
\setbeamertemplate{headline}{}

% スライドタイトルのスタイル設定
\setbeamertemplate{frametitle}{%
  \vspace*{-0.2cm}%
  \begin{beamercolorbox}[wd=\paperwidth,ht=1.2cm,dp=0.3cm,leftskip=0pt,rightskip=0pt]{frametitle}
    \hspace*{1em}\Huge\insertframetitle
  \end{beamercolorbox}
}

% セクションのスタイル設定
\newenvironment{sectionblock}[1]{%
  \begin{minipage}{\textwidth}%
    \textbf{\large #1}\par\vspace{0.5em}%
}{%
  \end{minipage}\vspace{1em}%
}

% セクションコンテンツ用マクロ
\newcommand{\sectioncontent}[2]{%
  \begin{sectionblock}{#1}%
    \begin{itemize}%
      #2%
    \end{itemize}%
  \end{sectionblock}%
}

\begin{document}

\begin{frame}
\frametitle{定義}

\sectioncontent{積分作用素}{
  \item Riemann積分の線形性と単調性を確認
  \item 関数空間 $C(I; \mathbb{R})$ を導入
  \item 積分作用素が連続線形作用素であることを示す
}

\sectioncontent{関数列と完備性}{
  \item $C(I; \mathbb{R})$ の完備性を証明
  \item 積分と極限の順序交換の条件を確認
  \item 積分作用素の連続性とその応用を示す
}

\sectioncontent{初期値問題}{
  \item 常微分方程式の初期値問題を積分方程式に変換
  \item Picardの逐次近似法による解の構成
  \item 完備性と作用素の性質を用いて一意性を示す
}

\end{frame}

\end{document}
