% !TEX program = lualatex
\documentclass[aspectratio=169]{beamer}
\usepackage{luatexja}
\usepackage{amsmath,amssymb}
\usepackage{graphicx}
\usepackage{hyperref}
\usepackage{fancybox}
\usepackage{tcolorbox}
\usetheme{Copenhagen}
\tcbuselibrary{skins}

% スライドのスタイル設定
\setbeamertemplate{navigation symbols}{}
\setbeamertemplate{footline}[frame number]
\setbeamertemplate{headline}{}

% スライドタイトルのスタイル設定
\setbeamertemplate{frametitle}{%
  \vspace*{-0.2cm}%
  \begin{beamercolorbox}[wd=\paperwidth,ht=1.2cm,dp=0.3cm,leftskip=0pt,rightskip=0pt]{frametitle}
    \hspace*{1em}\Huge\insertframetitle
  \end{beamercolorbox}
}

% セクションのスタイル設定
\newenvironment{sectionblock}[1]{%
  \begin{minipage}{\textwidth}%
    \textbf{\large #1}\par\vspace{0.5em}%
}{%
  \end{minipage}\vspace{1em}%
}

% セクションコンテンツ用マクロ
\newcommand{\sectioncontent}[2]{%
  \begin{sectionblock}{#1}%
    \begin{itemize}%
      #2%
    \end{itemize}%
  \end{sectionblock}%
}

\begin{document}

\begin{frame}{定義}

\sectioncontent{関数空間とノルム}{
  \item 有界閉区間 $I$ 上の連続関数全体の集合 $C(I; \mathbb{R})$ を考える.
  \item 関数 $f$ に対して $\|f\| = \sup_{x \in I} |f(x)|$ によりノルムを定める.
  \item この空間はノルム空間となり,完備性を持つ(バナッハ空間).
}

\sectioncontent{積分作用素}{
  \item Riemann積分は連続関数に対し定義され,線形性・単調性を持つ.
  \item 積分作用素 $T(f)(x) = \int_a^x f(t)\,dt$ は線形かつ連続.
  \item これは $C(I; \mathbb{R})$ 上の連続線形写像である.
}

\sectioncontent{初期値問題の形式}{
  \item 微分方程式:$\frac{d}{dx}Y(x) = f(x, Y(x)),\quad Y(a) = Y_0$
  \item 解の構成にはPicardの逐次近似法を用いる.
  \item 一意性はGronwallの不等式により示される.
}

\end{frame}

\end{document}
