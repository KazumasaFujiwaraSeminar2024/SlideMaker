\documentclass{article}
\usepackage{luatexja}
\usepackage{luatexja-fontspec}
\setmainjfont{IPAexMincho}
\usepackage{xcolor}
\usepackage{listings}
\usepackage{hyperref}

\lstset{
    basicstyle=\ttfamily,
    frame=single,
    breaklines=true,
    numbers=left,
    numberstyle=\tiny,
    backgroundcolor=\color{gray!10},
    rulesepcolor=\color{gray!10}
}

\lstdefinelanguage{json}{
    basicstyle=\normalsize\ttfamily,  
    numbers=left,
    numberstyle=\scriptsize, 
    stepnumber=1,
    numbersep=8pt,
    showstringspaces=false,
    breaklines=true,
    frame=lines,
    backgroundcolor=\color{gray!10},
    stringstyle=\color{red},
    keywordstyle=\color{blue}\bfseries,
    keywords={true, false, null}
}

\begin{document}

    \part{文書の見た目やスタイルをカスタマイズ}

    \begin{lstlisting}[language=bash]
        \lstset{
            basicstyle=\ttfamily,
            frame=single,
            breaklines=true,
            numbers=left,
            numberstyle=\tiny,
            backgroundcolor=\color{gray!10},
            rulesepcolor=\color{gray!10}  
        }
    \end{lstlisting}

        \begin{itemize}
            \item \texttt{basicstyle=\textbackslash normalsize\textbackslash ttfamily}: リスト内のテキストを通常サイズの固定幅フォント(タイプライターフォント)で表示
            \item \texttt{numbers=left}: 行番号を左側に表示
            \item \texttt{numberstyle=\textbackslash scriptsize}: 行番号のフォントサイズを小さく設定
            \item \texttt{stepnumber=1}: 全ての行に行番号をつける
            \item \texttt{numbersep=8pt}: 行番号とコードの間の距離を8ポイントに設定
            \item \texttt{showstringspaces=false}: 文字列内の空白を特別な記号で表示しないようにする
            \item \texttt{breaklines=true}: 長い行がページの端に達した場合に自動的に折り返す
            \item \texttt{frame=lines}: リストの上部と下部に線を引いて枠を作る
            \item \texttt{backgroundcolor=\textbackslash color\{gray!10\}}: リストの背景色を薄いグレー(10\%のグレー)で設定
            \item \texttt{stringstyle=\textbackslash color\{red\}}: 文字列を赤色で表示
            \item \texttt{keywordstyle=\textbackslash color\{blue\}\textbackslash bfseries}: キーワード(true, false, null)を青色で太字にする
            \item \texttt{keywords=\{true, false, null\}}: true, false, nullをJSONのキーワードとして認識し、キーワードスタイルを適用する
        \end{itemize}

    \part{JSONコードを挿入し、特定のスタイルを適用するための設定}

    \begin{lstlisting}[language=json]
        \lstdefinelanguage{json}{
            basicstyle=\normalsize\ttfamily,
            numbers=left,
            numberstyle=\scriptsize,
            stepnumber=1,
            numbersep=8pt,
            showstringspaces=false,
            breaklines=true,
            frame=lines,
            backgroundcolor=\color{gray!10},
            stringstyle=\color{red},
            keywordstyle=\color{blue}\bfseries,
            keywords={true, false, null}
        }
    \end{lstlisting}

    \begin{itemize}
        \item \texttt{basicstyle=\textbackslash{}normalsize\textbackslash{}ttfamily}: リスト内のテキストを通常サイズの固定幅フォント(タイプライターフォント)で表示します。
        \item \texttt{numbers=left}: 行番号を左側に表示します。
        \item \texttt{numberstyle=\textbackslash{}scriptsize}: 行番号のフォントサイズを小さく設定します。
        \item \texttt{stepnumber=1}: すべての行に行番号をつけます。
        \item \texttt{numbersep=8pt}: 行番号とコードの間の距離を8ポイントに設定します。
        \item \texttt{showstringspaces=false}: 文字列内の空白を特別な記号で表示しないようにします。
        \item \texttt{breaklines=true}: 長い行がページの端に達した場合に自動的に折り返します。
        \item \texttt{frame=lines}: リストの上部と下部に線を引いて枠を作ります。
        \item \texttt{backgroundcolor=\textbackslash{}color\{gray!10\}}: リストの背景色を薄いグレー(10\%のグレー)で設定します。
        \item \texttt{stringstyle=\textbackslash{}color\{red\}}: 文字列を赤色で表示します。
        \item \texttt{keywordstyle=\textbackslash{}color\{blue\}\textbackslash{}bfseries}: キーワード(true, false, null)を青色で太字にします。
        \item \texttt{keywords=\{true, false, null\}}: true, false, nullをJSONのキーワードとして認識し、キーワードスタイルを適用します。
    \end{itemize}

    \part{レポジトリへのCommit}

    \begin{enumerate}
        \item 以下のURLから先生のgithubに移動
        \url{https://github.com/KazumasaFUJIWARA/rgt}
        
        \item アカウントを作成
        \item メールアドレスを先生に送信して招待してもらう
        \item レポジトリ内のメールからメールを確認し、承諾する
        \item 招待されたサイトをデスクトップに追加しておく
    \end{enumerate}

\end{document}
