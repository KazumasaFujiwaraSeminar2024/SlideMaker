\documentclass[12pt]{beamer}
\usepackage{luatexja}
\usepackage{luatexja-fontspec}
\usepackage{amsmath, amssymb}
\usepackage{hyperref}
\usepackage{fancybox}
\usepackage{xcolor}
\usepackage{graphicx}

\usetheme{default}
\setbeamercolor{title}{bg=SkyBlue, fg=black}
\setbeamercolor{frametitle}{bg=SkyBlue, fg=black}
\setbeamerfont{frametitle}{size=\large}
\setbeamerfont{itemize/enumerate body}{size=\small}

\title{概要}
\author{石川 拓海}
\date{}

\begin{document}

\begin{frame}
\titlepage
\end{frame}

\begin{frame}
\frametitle{\doublebox{背景}}
\begin{itemize}
    \item 大学3年から解析学の基礎を学ぶ。
    \item 微分方程式への理解が深まる。
    \item 初期値問題の一意性に興味を持つ。
\end{itemize}
\end{frame}

\begin{frame}
\frametitle{\doublebox{目的}}
\begin{itemize}
    \item 常微分方程式の初期値問題における解の一意存在性を示す。
    \item 理論的な構成と証明を通じた理解深化。
\end{itemize}
\end{frame}

\begin{frame}
\frametitle{\doublebox{方法}}
\begin{itemize}
    \item Picardの逐次近似法を適用。
    \item 関数空間の完備性・積分作用素の有界性を活用。
    \item Gronwallの不等式により一意性を導出。
\end{itemize}
\end{frame}

\begin{frame}
\frametitle{\doublebox{結果}}
\begin{itemize}
    \item 有界閉区間上での解の一意性を証明。
    \item 解は積分方程式の不動点として構成可能。
\end{itemize}
\end{frame}

\end{document} 